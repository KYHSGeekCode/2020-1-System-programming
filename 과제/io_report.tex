\documentclass{article}
\usepackage{graphicx}
\usepackage{kotex}
\usepackage{lipsum}
\usepackage{listings}
\usepackage{graphicx}
\usepackage[a4paper, total={6in, 10in}]{geometry}
\usepackage{float}
\usepackage{xcolor}
\usepackage{color}
\usepackage{minted}
\usepackage{hyperref}
% \renewcommand{\thesection}{\arabic{section}}

\title{Fun with File Descriptors 레포트}
\date{}
\author{
  2019-13674\\
  양현서 \\
  바이오시스템소재학부\\
}
\lstset{
    basicstyle=\ttfamily\small,
    numberstyle=\footnotesize,
    numbers=left,
    frame=single,
    tabsize=2,
    title=\lstname,
    keywordstyle={\color{blue}}
    escapeinside={\%*}{*)},
    breaklines=true,
    breakatwhitespace=true,
    framextopmargin=2pt,
    framexbottommargin=2pt,
    extendedchars=false,
    inputencoding=utf8
}
\hypersetup{
    colorlinks,
    linkcolor={red!50!black},
    citecolor={blue!50!black},
    urlcolor={blue!80!black}
}

\makeatletter
\AtBeginEnvironment{minted}{\dontdofcolorbox}
\def\dontdofcolorbox{\renewcommand\fcolorbox[4][]{##4}}
\makeatother

\begin{document}
\maketitle
\section{Fun with file Descriptors(1)}
\begin{minted}[breaklines]{C}
  #include "csapp.h"

  int main(int argc, char * argv[]) {
    int fd1, fd2, fd3;
    char c1, c2, c3;
    char * fname = argv[1];
    fd1 = Open(fname, O_RDONLY, 0);
    fd2 = Open(fname, O_RDONLY, 0);
    fd3 = Open(fname, O_RDONLY, 0);
    Dup2(fd2, fd3);
    Read(fd1, & c1, 1);
    Read(fd2, & c2, 1);
    Read(fd3, & c3, 1);
    printf("c1 = %c, c2 = %c, c3 = %c\n", c1, c2, c3);
    return 0;
  }
\end{minted}

답: c1 = a, c2 = a, c3 = b

\section{Fun with file descriptors(2)}
\begin{minted}[breaklines]{C}
    #include "csapp.h"

    int main(int argc, char * argv[]) {
      int fd1;
      int s = getpid() & 0x1;
      char c1, c2;
      char * fname = argv[1];
      fd1 = Open(fname, O_RDONLY, 0);
      Read(fd1, & c1, 1);
      if (fork()) {
        /* Parent */
        sleep(s);
        Read(fd1, & c2, 1);
        printf("Parent: c1 = %c, c2 = %c\n", c1, c2);
      } else {
        /* Child */
        sleep(1 - s);
        Read(fd1, & c2, 1);
        printf("Child: c1 = %c, c2 = %c\n", c1, c2);
      }
      return 0;
    }
\end{minted}
답: \\
 parent pid가 홀수(s=1)일때: \\
 \hspace{10mm} Child: c1=a, c2=b \\
 \hspace{10mm} Parent: c1=a, c2=c \\ 
    parent pid가 짝수(s=0)일때: \\
    \hspace{10mm} Parent: c1=a, c2=b \\
    \hspace{10mm} Child: c1=a, c2=c
    

\section{Fun with file descriptors(3)}
\begin{minted}[breaklines]{C}
#include "csapp.h"

int main(int argc, char * argv[]) {
  int fd1, fd2, fd3;
  char * fname = argv[1];
  fd1 = Open(fname, O_CREAT | O_TRUNC | O_RDWR, S_IRUSR | S_IWUSR);
  Write(fd1, "pqrs", 4);
  fd3 = Open(fname, O_APPEND | O_WRONLY, 0);
  Write(fd3, "jklmn", 5);
  fd2 = dup(fd1); /* Allocates descriptor */
  Write(fd2, "wxyz", 4);
  Write(fd3, "ef", 2);
  return 0;
}
\end{minted}
답: pqrswxyznef

\end{document}