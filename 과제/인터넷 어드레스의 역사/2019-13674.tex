\documentclass{report}
\usepackage{graphicx}
\usepackage{kotex}
\usepackage{lipsum}
\usepackage{listings}
\usepackage{graphicx}
\usepackage[a4paper, total={6in, 10in}]{geometry}
\usepackage{float}
\usepackage{xcolor}
\usepackage{color}
\usepackage{minted}
\usepackage{hyperref}
\renewcommand{\thesection}{\arabic{section}}

\title{시스템 프로그래밍 7주차 수업 숙제\\
\large 인터넷 어드레스가 모자라기 시작해서 이를 해결하기 위하여 어떤 일들이 벌어졌는가}
\date{}
\author{
  2019-13674\\
  양현서 \\
  바이오시스템소재학부\\
}
\lstset{
    basicstyle=\ttfamily\small,
    numberstyle=\footnotesize,
    numbers=left,
    frame=single,
    tabsize=2,
    title=\lstname,
    keywordstyle={\color{blue}}
    escapeinside={\%*}{*)},
    breaklines=true,
    breakatwhitespace=true,
    framextopmargin=2pt,
    framexbottommargin=2pt,
    extendedchars=false,
    inputencoding=utf8
}
\hypersetup{
    colorlinks,
    linkcolor={red!50!black},
    citecolor={blue!50!black},
    urlcolor={blue!80!black}
}

\makeatletter
\AtBeginEnvironment{minted}{\dontdofcolorbox}
\def\dontdofcolorbox{\renewcommand\fcolorbox[4][]{##4}}
\makeatother

\begin{document}
\maketitle
\tableofcontents

\section{Introduction}
\paragraph{}
인터넷에 연결된 장치들이 서로 통신하기 위해서는 인터넷 시스템 상에서 각 호스트를 서로 구분해 낼 수 있게 해 주는 정보가 필요한데, 오늘날에는 MAC 어드레스와 IP 주소가 많이 쓰인다. MAC 어드레스는 물리적 주소로 48비트의 크기를 가진다. 물리적 네트워크 카드의 제조사와 일련번호 등으로 이루어진다. IP 주소는 논리적으로 정해지며, IP 주소 중 먼저 개발된 IPv4 IP 주소는 32비트(4바이트)의 크기를 가지고 있고, 최근에 나온 IPv6 IP 주소는 128비트(16바이트)의 크기를 가지고 있다. IPv4를 처음 만들었을 때는 인터넷에 연결된 장치들의 수가 별로 많지 않았고, 32비트 즉 약 43억가지의 IP 주소만 가지고도 모든 장치들을 구별할 수 있을 것이라 생각하였다. 그러나 인터넷에 연결된 장치들이 빠르게 증가함에 따라 이 주소들은 부족해지게 되었다. 

\section{DHCP}
DHCP는 Dynamic Host Configuration Protocol의 약자로 IP 주소가 필요한 장치가 DHCP 서버에서 동적으로 IP 주소를 할당받을 수 있게 함으로써 모든 장치가 IP 주소를 미리 받을 필요가 없고, 할당받은 IP 주소가 필요 없어지면 다시 반납할 수 있게 해주는 시스템이다. DHCP 서버가 할당을 받는다. 
처음에 장치가 DHCP 서버에서 IP 주소를 할당(lease)받는 과정은 다음과 같다.
\begin{enumerate}
  \item IP 주소를 할당받고자 하는 클라이언트가 DHCP 서버를 찾기 위해 자신의 MAC 주소를 포함하는 \lstinline{DHCPDISCOVER} 패킷을 브로드캐스트한다.
  \item DHCP 서버가 이 패킷을 받으면, 자신이 DHCP 서버임을 알리기 위해 해당 MAC 주소를 이용해 \lstinline{DHCPOFFER} 패킷을 브로드캐스트한다.
  \item 클라이언트는 DHCP 서버들에서 보내는 패킷들 중 먼저 수신한 패킷을 처리하고, \lstinline{DHCPREQUEST} 패킷을 브로드캐스트함으로써 해당 DHCP 서버에 IP 주소를 요청함과 동시에, 다른 DHCP 서버들에게 클라이언트가 적당한 DHCP 서버를 이미 찾았음을 알린다.
  \item DHCP 서버는 자신이 가지고 있는 리스트에서 할당되지 않은 IP 주소를 검색하여, DNS 서버 주소 등 여러 가지 부가 정보와 함께 해당 클라이언트에게 응답해준다.
\end{enumerate}
DHCP로 할당받은 IP주소에는 유효기간이 존재한다. DHCP 서버에서 IP 주소를 할당할 때 유효기간도 같이 부여하는데, 이 유효기간 내에 갱신이 이루어지지 않으면 그 IP 주소는 반환되게 된다. 또는 클라이언트가 더 이상 IP 주소가 필요 없어질 경우 직접 반환하게 된다. 이렇게 할당받은 IP를 DHCP 서버에 반환하고, 필요한 기기만 다시 할당하는 방식을 사용함으로써 고정 IP를 사용할 때보다 유휴 IP 활용률이 높아진다.

\section{Subnetting}
서브넷팅이란 네트워크를 작게 쪼개어 사용하는 것을 말한다. 이렇게 하면 일부만 사용되는 IP 대역을 세분화하여 더 효율적으로 IP주소를 배분할 수 있다. 극단적인 예시로 class A IP 주소를 장치가 많지 않은 가정집이 사용한다고 하면, 몇개의 IP주소만 사용하고 나머지 주소들은 사용되지 않은 채 버려질 것이다. 그런데 그 IP 주소를 서브넷 마스크와 같은 것을 이용해 서브넷으로 세분화하면, IP주소를 더욱 효율적으로 사용할 수 있게 된다. 또한, 서브넷을 이용하면 브로드캐스트 패킷의 전달 범위를 줄일 수 있어, 네트워크의 부하를 줄이는데도 도움이 된다.
서브넷 마스크라는, 상위비트부터 연속된 1로 이루어진 마스크를 통해 어떤 IP 주소의 서브넷 ID와 호스트 ID를 알아낼 수 있다. 서브넷 마스크와 AND하여 얻어낸 것이 서브넷 ID, 나머지 하위 비트가 호스트 ID를 나타내게 하면, Class A, B, C로 나누는 것보다 더 세분화된 구분이 가능하여 사용하지 않는 IP대역을 다른 서브넷으로 이용할 수 있게 할 수 있는 것이다.

\section{NAT/PAT}
NAT(Network address translation)과 PAT(Port address translation)은 라우터 장치에서 패킷의 IP주소와 포트 정보를 변경하는 것이다. 이것을 통해 IPv4 주소의 고갈을 늦출 수 있다. 이것을 극단으로 이용한 것이 Masquerading(NPAT)인데, 단순한 IPv4만을 이용하면 최대로 잡아서 $2^{32}$ 가지의 주소밖에 표현할 수 없지만, 포트 번호는 $2^{16}$가지가 있으므로 그만큼의 IP 주소를 절약할 수 있다.
이를 위하여 라우터 같은 장비에서 테이블을 만들고, 내부 네트워크의 IP주소와 공유기 자신의 공인 IP 주소를 이용하여 내부 네트워크의 호스트들을 구별할 수 있게 하는 테이블을 만든다. 이렇게 하면 외부의 입장에서는 내부의 네트워크에 연결된 호스트들을 직접 보지 않게 할 수 있으며, 내부 네트워크에서 포트 번호로 멀티플렉싱을 하면서 내부망에서는 내부적으로만 사용되는 사설 IP주소를 사용할 경우 외부의 입장에서는 IPv4 주소를 할당받아가는 양이 현저하게 줄어들게 느끼게 할 수 있다.
\\
단점으로는 서버와 같이 고정된 포트에서 외부의 접속을 기다리는 경우에는, NAT를 사용하였을 때 포트 번호가 바뀔 수 있는 특징에 의하여 제대로 된 작동을 하지 않을 수 있다는 것이다. 따라서 포트 포워딩 설정 등을 통해 이러한 특별한 포트로 오는 패킷들을 대응하는 호스트로 전송하도록 설정해 주는 것이 필요하다는 번거로움이 생긴다.

\section{IPv6}
IP 주소 부족의 궁극적 해결책은 IP 주소를 표현하는데 사용되는 비트수를 늘리는 것이었다. 4바이트(32비트)에서 16바이트(128비트)로 늘어났다. IPv4 다음 프로토콜이 왜 IPv5가 아니라 IPv6인지도 찾아보았는데, 비디오/오디오 스트리밍을 위해 이미 IPv5라는 이름의 프로토콜이 있어서 그 다음 번호인 IPv6을 사용한 것이었다. IPv6은 \lstinline{2001:0db8:0000:0000:0000:ff00:0042:8329}와 같이 128비트의 숫자를 16진수로 4자리씩 끊어서 나타내며, \lstinline{::ffff:0:0/96} 의 어드레스는 IPv4의 어드레스에 대응하게 만들었다.

\section{Conclusion}
아직 네트워크 시스템이 그렇게 발달하지 않았던 1980년대에는 미래에 그렇게 많은 장치들이 인터넷에 접속할 것을 고려하지 못하고 32비트 정수를 사용하여 IP 주소를 나타내게 하였고, 이것이 현재 문제가 되고 있는 것이다. 이것은 마치 유닉스 시간이 32비트이기 때문에 2038년에 문제가 생기게 되는 것과 유사하다고 볼 수 있다. 2038년 문제도 64비트로 정보의 크기를 늘려 해결하듯이, IPv4 고갈 문제도 128비트를 사용하는 IPv6을 이용함으로써 해결할 수 있다. 그러나 이미 IPv4가 많이 퍼져 있어서 바로 IPv6으로 넘어가는 것에는 무리가 따르기 때문에, 위에서 언급한 여러 가지 방법이 현재 IPv4 address 고갈 문제를 해결하기 위해 고안된 것이다. 당장 모든 네트워크 통신 방식을 IPv6으로 바꾸기 쉽지 않은 만큼, 이러한 방법들을 공부해 보는 것도 의미가 있다고 생각했다.

\bibliographystyle{abbrv}
\begin{thebibliography}{9}
  \bibitem{knuthwebsite} 
  Understanding DHCP discovery specific subnet,
  \\\texttt{https://superuser.com/a/811523/964075}
  
  \bibitem{knuthwebsite} 
  IPv6 address,
  \\\texttt{https://en.wikipedia.org/wiki/IPv6\_address}
  
  \bibitem{knuthwebsite} 
  염헌영 교수님의 강의자료
  \\\texttt{}
  

\end{thebibliography}
\end{document}