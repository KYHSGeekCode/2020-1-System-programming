\documentclass{report}
\usepackage{graphicx}
\usepackage{kotex}
\usepackage{lipsum}
\usepackage{listings}
\usepackage{graphicx}
\usepackage[a4paper, total={6in, 10in}]{geometry}
\usepackage{float}
\usepackage{xcolor}
\usepackage{color}
\usepackage{minted}
\usepackage{hyperref}
\renewcommand{\thesection}{\arabic{section}}

\title{시스템 프로그래밍 과제 1 linklab 레포트 \\
\large Linker Lab: Memtrace }
\date{}
\author{
  2019-13674\\
  양현서 \\
  바이오시스템소재학부\\
}
\lstset{
    basicstyle=\ttfamily\small,
    numberstyle=\footnotesize,
    numbers=left,
    frame=single,
    tabsize=2,
    title=\lstname,
    keywordstyle={\color{blue}}
    escapeinside={\%*}{*)},
    breaklines=true,
    breakatwhitespace=true,
    framextopmargin=2pt,
    framexbottommargin=2pt,
    extendedchars=false,
    inputencoding=utf8
}
\hypersetup{
    colorlinks,
    linkcolor={red!50!black},
    citecolor={blue!50!black},
    urlcolor={blue!80!black}
}

\makeatletter
\AtBeginEnvironment{minted}{\dontdofcolorbox}
\def\dontdofcolorbox{\renewcommand\fcolorbox[4][]{##4}}
\makeatother

\begin{document}
\maketitle
\tableofcontents

\section{Introduction}
\paragraph{}
 수업 시간에 Library interpositioning을 배웠다. 이것은 어떤 프로그램이 실행될 때 사용되는 외부 공유 라이브러리의 함수 호출을 중간에서 가로채 임의의 다른 함수를 실행할 수 있게 하는 기법이다. Compile time, Link time, Load/Run time에 가능한데, 이번 lab에서는 Load/Run time에 사용하였다.
\paragraph{}
 구체적으로는 \lstinline{malloc}, \lstinline{free}, \lstinline{calloc}, 그리고 \lstinline{realloc}을 interpositioning하여 테스트 프로그램들의 메모리 할당과 해제를 추적하고, 해제되지 않은 메모리와 그 메모리를 할당한 위치 등과 같은 유용한 정보도 출력하는 라이브러리를 만든다.

\section{Part 1: Tracing Dynamic Memory Allocation}
Part 1에서는 테스트 프로그램들의 \lstinline{malloc}, \lstinline{free}, \lstinline{calloc}, 그리고 \lstinline{realloc} 함수들의 호출과 그 결과값을 출력하고, 할당된 메모리와 한번 호출당 할당된 메모리의 평균을 출력한다.
part 1은 library interpositioning 실습의 몸풀기라고 볼 수 있다. 처음에는 \lstinline{malloc}과 \lstinline{free} 만이 대상인 줄 잘못 이해하였으나 나중에 \lstinline{calloc}과 \lstinline{realloc}도 구현하였다. handout에 나온 결과 예시를 보면 \lstinline{malloc}과 \lstinline{free}의 호출 정보와 결과값을 화면에 나타내고, 마지막에 총 할당 정보를 표시한다. 처음에는 \lstinline{mlog} 함수를 직접 호출하여 화면과 비슷하게 출력하게 하려 하였으나, 곧 memlog.h를 다시 살펴보고 나서 \lstinline{LOG_MALLOC}등과 같은 매크로들을 이용하면 된다는 것을 알게 되었다.


library interpositioning을 성공적으로 수행하기 위해, \lstinline{init} 함수에 실제  \lstinline{malloc}, \lstinline{free}, \lstinline{calloc}, 그리고 \lstinline{realloc} 함수들에 대한 포인터를 초기화하는 루틴을 넣었다.

\inputminted[firstline=86,lastline=105, linenos]{C}{2019-13674.224747/part1/memtrace.c}
여기서 \lstinline{RTLD_NEXT}를 사용하여 현재 라이브러리가 아닌 다음 라이브러리에서 함수 심볼들을 찾으라고 링커에게 명령하였다.
\\
part 1의 가로채어 대신 실행되는 함수들은 아래와 같이 작성하였다. 
\inputminted[firstline=35,lastline=42, linenos]{C}{2019-13674.224747/part1/memtrace.c}
간단하게 총 할당 바이트 수를 나타내는 \lstinline{n_mallocb}에 \lstinline{size}만큼 더하고 \lstinline{n_malloc}을 1 증가시키고 끝난다.
\inputminted[firstline=44,lastline=48, linenos]{C}{2019-13674.224747/part1/memtrace.c}
\lstinline{free}의 경우는 더 간단하게 \lstinline{LOG_FREE}만 이용하면 된다.
\inputminted[firstline=50,lastline=59, linenos]{C}{2019-13674.224747/part1/memtrace.c}
\lstinline{calloc}의 경우는 할당되는 최종 바이트 크기가 \lstinline{nmemb}$\times$\lstinline{size} 인 것만 주의하면 \lstinline{malloc}과 비슷하다. \lstinline{if(resultP)} 부분은 할당의 성공을 체크하는 부분인데, \lstinline{calloc}과 \lstinline{realloc}을 구현하면서 오류 확인 목적으로 넣었다.
\inputminted[firstline=60,lastline=69, linenos]{C}{2019-13674.224747/part1/memtrace.c}
\lstinline{realloc}도 아직까지는 특별하게 처리할 것이 없다.

\inputminted[firstline=113,lastline=124, linenos]{C}{2019-13674.224747/part1/memtrace.c}
라이브러리가 언로드될 때 호출되는 \lstinline{fini} 함수에서는 \lstinline{LOG_STATISTICS} 매크로를 이용해 총 메모리 할당량과 평균 메모리 할당량을 표시한다. 평균 메모리 할당량을 구할 때 처음에는 \lstinline{n_malloc}으로만 나누었다가, \lstinline{n_malloc+n_calloc+n_realloc}으로 나누는 것으로 수정하였다.

\subsection{테스트 결과}
\subsubsection{test1}
\begin{minted}[breaklines]{console}
user102@SystemProgramming:~/handout/part1$ make run test1
cc -I. -I ../utils -o libmemtrace.so -shared -fPIC memtrace.c ../utils/memlog.c ../utils/memlist.c callinfo.c -ldl -lunwind
[0001] Memory tracer started.
[0002]           (nil) : malloc( 1024 ) = 0x16e1060
[0003]           (nil) : malloc( 32 ) = 0x16e1470
[0004]           (nil) : malloc( 1 ) = 0x16e14a0
[0005]           (nil) : free( 0x16e14a0 )
[0006]           (nil) : free( 0x16e1470 )
[0007]
[0008] Statistics
[0009]   allocated_total      1057
[0010]   allocated_avg        352
[0011]   freed_total          0
[0012]
[0013] Memory tracer stopped.
\end{minted}
아직 함수 호출자 부분은 nil로 나오고, \lstinline{freed_total}도 0인 것을 볼 수 있다.
\subsubsection{test2}
\begin{minted}[breaklines]{console}
  user102@SystemProgramming:~/handout/part1$ make run test2
  cc -I. -I ../utils -o libmemtrace.so -shared -fPIC memtrace.c ../utils/memlog.c ../utils/memlist.c callinfo.c -ldl -lunwind
  [0001] Memory tracer started.
  [0002]           (nil) : malloc( 1024 ) = 0x24c3060
  [0003]           (nil) : free( 0x24c3060 )
  [0004]
  [0005] Statistics
  [0006]   allocated_total      1024
  [0007]   allocated_avg        1024
  [0008]   freed_total          0
  [0009]
  [0010] Memory tracer stopped.
\end{minted}

\subsubsection{test3}
\begin{minted}[breaklines]{console}
  user102@SystemProgramming:~/handout/part1$ make run test3
  cc -I. -I ../utils -o libmemtrace.so -shared -fPIC memtrace.c ../utils/memlog.c ../utils/memlist.c callinfo.c -ldl -lunwind
[0001] Memory tracer started.
[0002]           (nil) : malloc( 42213 ) = 0x1d51060
[0003]           (nil) : malloc( 22581 ) = 0x1d5b550
[0004]           (nil) : calloc( 1 , 727 ) = 0x1d60d90
[0005]           (nil) : calloc( 1 , 64048 ) = 0x1d61070
[0006]           (nil) : calloc( 1 , 50720 ) = 0x1d70ab0
[0007]           (nil) : malloc( 43080 ) = 0x1d7d0e0
[0008]           (nil) : calloc( 1 , 61740 ) = 0x1d87930
[0009]           (nil) : malloc( 37447 ) = 0x1d96a70
[0010]           (nil) : calloc( 1 , 37103 ) = 0x1d9fcc0
[0011]           (nil) : calloc( 1 , 59380 ) = 0x1da8dc0
[0012]           (nil) : free( 0x1da8dc0 )
[0013]           (nil) : free( 0x1d9fcc0 )
[0014]           (nil) : free( 0x1d96a70 )
[0015]           (nil) : free( 0x1d87930 )
[0016]           (nil) : free( 0x1d7d0e0 )
[0017]           (nil) : free( 0x1d70ab0 )
[0018]           (nil) : free( 0x1d61070 )
[0019]           (nil) : free( 0x1d60d90 )
[0020]           (nil) : free( 0x1d5b550 )
[0021]           (nil) : free( 0x1d51060 )
[0022]
[0023] Statistics
[0024]   allocated_total      419039
[0025]   allocated_avg        41903
[0026]   freed_total          0
[0027]
[0028] Memory tracer stopped.
\end{minted}

\subsubsection{test4}
\begin{minted}[breaklines]{console}
  user102@SystemProgramming:~/handout/part1$ make run test4
cc -I. -I ../utils -o libmemtrace.so -shared -fPIC memtrace.c ../utils/memlog.c ../utils/memlist.c callinfo.c -ldl -lunwind
[0001] Memory tracer started.
[0002]           (nil) : malloc( 1024 ) = 0x107c060
[0003]           (nil) : free( 0x107c060 )
[0004]           (nil) : free( 0x107c060 )
*** Error in `../test/test4': double free or corruption (top): 0x000000000107c060 ***
[0005]           (nil) : malloc( 36 ) = 0x7f4b8c0008c0
[0006]           (nil) : calloc( 1182 , 1 ) = 0x7f4b8c0008f0
[0007]           (nil) : malloc( 36 ) = 0x7f4b8c000da0
[0008]           (nil) : malloc( 56 ) = 0x7f4b8c000dd0
[0009]           (nil) : calloc( 15 , 24 ) = 0x7f4b8c000e10
======= Backtrace: =========
/lib/x86_64-linux-gnu/libc.so.6(+0x777e5)[0x7f4b930647e5]
/lib/x86_64-linux-gnu/libc.so.6(+0x8037a)[0x7f4b9306d37a]
/lib/x86_64-linux-gnu/libc.so.6(cfree+0x4c)[0x7f4b9307153c]
./libmemtrace.so(free+0x39)[0x7f4b933b7d8a]
../test/test4[0x40048e]
/lib/x86_64-linux-gnu/libc.so.6(__libc_start_main+0xf0)[0x7f4b9300d830]
../test/test4[0x4004c9]
======= Memory map: ========
00400000-00401000 r-xp 00000000 ca:01 657555                             /home/user102/handout/test/test4
00600000-00601000 r--p 00000000 ca:01 657555                             /home/user102/handout/test/test4
00601000-00602000 rw-p 00001000 ca:01 657555                             /home/user102/handout/test/test4
0107c000-0109d000 rw-p 00000000 00:00 0                                  [heap]
7f4b8c000000-7f4b8c021000 rw-p 00000000 00:00 0
7f4b8c021000-7f4b90000000 ---p 00000000 00:00 0
7f4b92bd3000-7f4b92be9000 r-xp 00000000 ca:01 2097679                    /lib/x86_64-linux-gnu/libgcc_s.so.1
... (중략)
7f4b933b1000-7f4b933b3000 rw-p 001c4000 ca:01 2097653                    /lib/x86_64-linux-gnu/libc-2.23.so
7f4b933b3000-7f4b933b7000 rw-p 00000000 00:00 0
7f4b933b7000-7f4b933b9000 r-xp 00000000 ca:01 657531                     /home/user102/handout/part1/libmemtrace.so
7f4b933b9000-7f4b935b8000 ---p 00002000 ca:01 657531                     /home/user102/handout/part1/libmemtrace.so
7f4b935b8000-7f4b935b9000 r--p 00001000 ca:01 657531                     /home/user102/handout/part1/libmemtrace.so
7f4b935b9000-7f4b935ba000 rw-p 00002000 ca:01 657531                     /home/user102/handout/part1/libmemtrace.so
7f4b935ba000-7f4b935e0000 r-xp 00000000 ca:01 2097629                    /lib/x86_64-linux-gnu/ld-2.23.so
7f4b937d5000-7f4b937d8000 rw-p 00000000 00:00 0
7f4b937dd000-7f4b937df000 rw-p 00000000 00:00 0
7f4b937df000-7f4b937e0000 r--p 00025000 ca:01 2097629                    /lib/x86_64-linux-gnu/ld-2.23.so
7f4b937e0000-7f4b937e1000 rw-p 00026000 ca:01 2097629                    /lib/x86_64-linux-gnu/ld-2.23.so
7f4b937e1000-7f4b937e2000 rw-p 00000000 00:00 0
7ffd8755e000-7ffd8757f000 rw-p 00000000 00:00 0                          [stack]
7ffd87581000-7ffd87584000 r--p 00000000 00:00 0                          [vvar]
7ffd87584000-7ffd87586000 r-xp 00000000 00:00 0                          [vdso]
ffffffffff600000-ffffffffff601000 r-xp 00000000 00:00 0                  [vsyscall]
Aborted (core dumped)
Makefile:37: recipe for target 'run' failed
make: *** [run] Error 134
\end{minted}
아직 double free와 illegal free 처리를 하지 않기 때문에 크래시되는 것을 볼 수 있다. 

\subsubsection{test5}
\begin{minted}[breaklines]{console}
  user102@SystemProgramming:~/handout/part1$ make run test5
cc -I. -I ../utils -o libmemtrace.so -shared -fPIC memtrace.c ../utils/memlog.c ../utils/memlist.c callinfo.c -ldl -lunwind
[0001] Memory tracer started.
[0002]           (nil) : malloc( 10 ) = 0x154c060
[0003]           (nil) : realloc( 0x154c060 , 100 ) = 0x154c060
[0004]           (nil) : realloc( 0x154c060 , 1000 ) = 0x154c060
[0005]           (nil) : realloc( 0x154c060 , 10000 ) = 0x154c060
[0006]           (nil) : realloc( 0x154c060 , 100000 ) = 0x154c060
[0007]           (nil) : free( 0x154c060 )
[0008]
[0009] Statistics
[0010]   allocated_total      111110
[0011]   allocated_avg        22222
[0012]   freed_total          0
[0013]
[0014] Memory tracer stopped.
\end{minted}

\section{Part 2: Tracing Unfreed Memory}
Part 2에서는 Part 1의 정보와 더불어 해제되는 메모리에 대한 정보도 출력하며, 해제되지 않은 메모리의 양과 할당 위치에 대한 정보를 제공한다. 이제부터는 part 1과 다르게 각 메모리 할당을 개개 함수 스코프를 넘어 추적해야 한다. 그러기 위하여 자료구조가 필요한데, 마침 utils/memlist.h를 이용하여 memlist.c에 구현되어 있는 링크드 리스트 자료구조를 이용할 수 있었다.
\inputminted[firstline=35,lastline=43, linenos]{C}{2019-13674.224747/part2/memtrace.c}
part 2에서 추가된 코드는 \lstinline{alloc} 함수를 호출하는 것이다. \lstinline{alloc} 함수를 호출하면 자동으로 새로운 item을 생성하거나 이미 존재할 경우 reference count를 증가시켜 준다.

\inputminted[firstline=45,lastline=51, linenos]{C}{2019-13674.224747/part2/memtrace.c}
\lstinline{free}에는 \lstinline{dealloc}을 이용하여 reference count를 감소시켜준다. 또 이 함수의 리턴값인 그 \lstinline{item}의 주소를 이용하여 해당 주소 메모리 영역의 크기를 \lstinline{n_freeb}에 더한다.

\inputminted[firstline=53,lastline=63, linenos]{C}{2019-13674.224747/part2/memtrace.c}
\lstinline{calloc} 은 앞서 언급하였듯이 할당되는 최종 바이트 크기가 \lstinline{nmemb}$\times$\lstinline{size} 인 것만 주의하면 \lstinline{malloc}과 동일한 내용이다.

\inputminted[firstline=64,lastline=76, linenos]{C}{2019-13674.224747/part2/memtrace.c}
\lstinline{realloc}은 앞의 \lstinline{malloc}과 \lstinline{free}의 내용을 둘 다 가지고 있다.

\inputminted[firstline=120,lastline=142, linenos]{C}{2019-13674.224747/part2/memtrace.c}
\lstinline{fini}는 non freed 블록이 존재할 경우 그에 대한 정보를 출력한다.

\subsection{테스트 결과}
\subsubsection{test1}
\begin{minted}[breaklines]{console}
  user102@SystemProgramming:~/handout/part2$ make run test1
  cc -I. -I ../utils -o libmemtrace.so -shared -fPIC memtrace.c ../utils/memlog.c ../utils/memlist.c callinfo.c -ldl -lunwind
  [0001] Memory tracer started.
  [0002]           (nil) : malloc( 1024 ) = 0x1d20060
  [0003]           (nil) : malloc( 32 ) = 0x1d204c0
  [0004]           (nil) : malloc( 1 ) = 0x1d20540
  [0005]           (nil) : free( 0x1d20540 )
  [0006]           (nil) : free( 0x1d204c0 )
  [0007]
  [0008] Statistics
  [0009]   allocated_total      1057
  [0010]   allocated_avg        352
  [0011]   freed_total          33
  [0012]
  [0013] Non-deallocated memory blocks
  [0014]   block              size       ref cnt   caller
  [0015]   0x1d20060          1024       1         ???:0
  [0016]
  [0017] Memory tracer stopped.
\end{minted}
test1의 소스코드를 보면 첫 번째 \lstinline{malloc} 호출로 할당한 메모리를 해제하지 않는 것을 볼 수 있는데, 이 테스트 결과도 그것을 잘 나타내고 있는 것을 볼 수 있다.
\subsubsection{test2}
\begin{minted}[breaklines]{console}
  user102@SystemProgramming:~/handout/part2$ make run test2
  cc -I. -I ../utils -o libmemtrace.so -shared -fPIC memtrace.c ../utils/memlog.c ../utils/memlist.c callinfo.c -ldl -lunwind
  [0001] Memory tracer started.
  [0002]           (nil) : malloc( 1024 ) = 0x1298060
  [0003]           (nil) : free( 0x1298060 )
  [0004]
  [0005] Statistics
  [0006]   allocated_total      1024
  [0007]   allocated_avg        1024
  [0008]   freed_total          1024
  [0009]
  [0010] Memory tracer stopped.
\end{minted}
1024바이트의 메모리를 할당받고 그것을 그대로 해제하는 것이 잘 나타나 있다.

\subsubsection{test3}
\begin{minted}[breaklines]{console}
  user102@SystemProgramming:~/handout/part2$ make run test3
  cc -I. -I ../utils -o libmemtrace.so -shared -fPIC memtrace.c ../utils/memlog.c ../utils/memlist.c callinfo.c -ldl -lunwind
[0001] Memory tracer started.
[0002]           (nil) : calloc( 1 , 43776 ) = 0x19b3060
[0003]           (nil) : calloc( 1 , 190 ) = 0x19bdbc0
[0004]           (nil) : calloc( 1 , 13781 ) = 0x19bdce0
[0005]           (nil) : calloc( 1 , 43393 ) = 0x19c1310
[0006]           (nil) : calloc( 1 , 58232 ) = 0x19cbcf0
[0007]           (nil) : malloc( 39935 ) = 0x19da0c0
[0008]           (nil) : malloc( 31759 ) = 0x19e3d20
[0009]           (nil) : malloc( 30749 ) = 0x19eb990
[0010]           (nil) : calloc( 1 , 33536 ) = 0x19f3210
[0011]           (nil) : calloc( 1 , 36193 ) = 0x19fb570
[0012]           (nil) : free( 0x19fb570 )
[0013]           (nil) : free( 0x19f3210 )
[0014]           (nil) : free( 0x19eb990 )
[0015]           (nil) : free( 0x19e3d20 )
[0016]           (nil) : free( 0x19da0c0 )
[0017]           (nil) : free( 0x19cbcf0 )
[0018]           (nil) : free( 0x19c1310 )
[0019]           (nil) : free( 0x19bdce0 )
[0020]           (nil) : free( 0x19bdbc0 )
[0021]           (nil) : free( 0x19b3060 )
[0022]
[0023] Statistics
[0024]   allocated_total      331544
[0025]   allocated_avg        33154
[0026]   freed_total          331544
[0027]
[0028] Memory tracer stopped.
\end{minted}

\subsubsection{test4}
\begin{minted}[breaklines]{console}
  user102@SystemProgramming:~/handout/part2$ make run test4
  cc -I. -I ../utils -o libmemtrace.so -shared -fPIC memtrace.c ../utils/memlog.c ../utils/memlist.c callinfo.c -ldl -lunwind
  [0001] Memory tracer started.
  [0002]           (nil) : malloc( 1024 ) = 0xf90060
  [0003]           (nil) : free( 0xf90060 )
  [0004]           (nil) : free( 0xf90060 )
  *** Error in `../test/test4': double free or corruption (!prev): 0x0000000000f90060 ***
  [0005]           (nil) : malloc( 36 ) = 0x7f85a80008c0
  [0006]           (nil) : calloc( 1182 , 1 ) = 0x7f85a8000940
  [0007]           (nil) : malloc( 36 ) = 0x7f85a8000e40
  [0008]           (nil) : malloc( 56 ) = 0x7f85a8000ec0
  [0009]           (nil) : calloc( 15 , 24 ) = 0x7f85a8000f50
  ======= Backtrace: =========
  /lib/x86_64-linux-gnu/libc.so.6(+0x777e5)[0x7f85aef9f7e5]
  /lib/x86_64-linux-gnu/libc.so.6(+0x8037a)[0x7f85aefa837a]
  /lib/x86_64-linux-gnu/libc.so.6(cfree+0x4c)[0x7f85aefac53c]
  ./libmemtrace.so(free+0x6c)[0x7f85af2f2e2b]
  ../test/test4[0x40048e]
  /lib/x86_64-linux-gnu/libc.so.6(__libc_start_main+0xf0)[0x7f85aef48830]
  ../test/test4[0x4004c9]
  ======= Memory map: ========
  00400000-00401000 r-xp 00000000 ca:01 657555                             /home/user102/handout/test/test4
  00600000-00601000 r--p 00000000 ca:01 657555                             /home/user102/handout/test/test4
  00601000-00602000 rw-p 00001000 ca:01 657555                             /home/user102/handout/test/test4
  00f90000-00fb1000 rw-p 00000000 00:00 0                                  [heap]
  7f85a8000000-7f85a8021000 rw-p 00000000 00:00 0
  7f85a8021000-7f85ac000000 ---p 00000000 00:00 0
  7f85aeb0e000-7f85aeb24000 r-xp 00000000 ca:01 2097679                    /lib/x86_64-linux-gnu/libgcc_s.so.1
  7f85aeb24000-7f85aed23000 ---p 00016000 ca:01 2097679                    /lib/x86_64-linux-gnu/libgcc_s.so.1
  7f85aed23000-7f85aed24000 rw-p 00015000 ca:01 2097679                    /lib/x86_64-linux-gnu/libgcc_s.so.1
  7f85aed24000-7f85aed27000 r-xp 00000000 ca:01 2097667                    /lib/x86_64-linux-gnu/libdl-2.23.so
  7f85aed27000-7f85aef26000 ---p 00003000 ca:01 2097667                    /lib/x86_64-linux-gnu/libdl-2.23.so
  7f85aef26000-7f85aef27000 r--p 00002000 ca:01 2097667                    /lib/x86_64-linux-gnu/libdl-2.23.so
  7f85aef27000-7f85aef28000 rw-p 00003000 ca:01 2097667                    /lib/x86_64-linux-gnu/libdl-2.23.so
  7f85aef28000-7f85af0e8000 r-xp 00000000 ca:01 2097653                    /lib/x86_64-linux-gnu/libc-2.23.so
  7f85af0e8000-7f85af2e8000 ---p 001c0000 ca:01 2097653                    /lib/x86_64-linux-gnu/libc-2.23.so
  7f85af2e8000-7f85af2ec000 r--p 001c0000 ca:01 2097653                    /lib/x86_64-linux-gnu/libc-2.23.so
  7f85af2ec000-7f85af2ee000 rw-p 001c4000 ca:01 2097653                    /lib/x86_64-linux-gnu/libc-2.23.so
  7f85af2ee000-7f85af2f2000 rw-p 00000000 00:00 0
  7f85af2f2000-7f85af2f4000 r-xp 00000000 ca:01 657556                     /home/user102/handout/part2/libmemtrace.so
  7f85af2f4000-7f85af4f4000 ---p 00002000 ca:01 657556                     /home/user102/handout/part2/libmemtrace.so
  7f85af4f4000-7f85af4f5000 r--p 00002000 ca:01 657556                     /home/user102/handout/part2/libmemtrace.so
  7f85af4f5000-7f85af4f6000 rw-p 00003000 ca:01 657556                     /home/user102/handout/part2/libmemtrace.so
  7f85af4f6000-7f85af51c000 r-xp 00000000 ca:01 2097629                    /lib/x86_64-linux-gnu/ld-2.23.so
  7f85af711000-7f85af714000 rw-p 00000000 00:00 0
  7f85af719000-7f85af71b000 rw-p 00000000 00:00 0
  7f85af71b000-7f85af71c000 r--p 00025000 ca:01 2097629                    /lib/x86_64-linux-gnu/ld-2.23.so
  7f85af71c000-7f85af71d000 rw-p 00026000 ca:01 2097629                    /lib/x86_64-linux-gnu/ld-2.23.so
  7f85af71d000-7f85af71e000 rw-p 00000000 00:00 0
  7ffe57c0d000-7ffe57c2e000 rw-p 00000000 00:00 0                          [stack]
  7ffe57da8000-7ffe57dab000 r--p 00000000 00:00 0                          [vvar]
  7ffe57dab000-7ffe57dad000 r-xp 00000000 00:00 0                          [vdso]
  ffffffffff600000-ffffffffff601000 r-xp 00000000 00:00 0                  [vsyscall]
  Aborted (core dumped)
  Makefile:37: recipe for target 'run' failed
  make: *** [run] Error 134
\end{minted}
아직 illegal free와 double free를 처리하지 않기 때문에 여전히 크래시가 발생하는 것을 볼 수 있다.
\subsubsection{test5}
\begin{minted}[breaklines]{console}
  user102@SystemProgramming:~/handout/part2$ make run test5
  cc -I. -I ../utils -o libmemtrace.so -shared -fPIC memtrace.c ../utils/memlog.c ../utils/memlist.c callinfo.c -ldl -lunwind
[0001] Memory tracer started.
[0002]           (nil) : malloc( 10 ) = 0x1a75060
[0003]           (nil) : realloc( 0x1a75060 , 100 ) = 0x1a750d0
[0004]           (nil) : realloc( 0x1a750d0 , 1000 ) = 0x1a75190
[0005]           (nil) : realloc( 0x1a75190 , 10000 ) = 0x1a755d0
[0006]           (nil) : realloc( 0x1a755d0 , 100000 ) = 0x1a755d0
[0007]           (nil) : free( 0x1a755d0 )
[0008]
[0009] Statistics
[0010]   allocated_total      111110
[0011]   allocated_avg        22222
[0012]   freed_total          111110
[0013]
[0014] Memory tracer stopped.
\end{minted}  
\lstinline{realloc} 으로 해제되는 메모리도 제대로 추적되는 것을 볼 수 있다.

\section{Part 3: Pinpointing Call Locations}
Part 3에서는 Part 2의 정보에 더불어, \href{https://github.com/libunwind/libunwind}{libunwind} 라이브러리를 이용하여 이 메모리 관리 함수들의 호출 위치를 추적한다. 이번 파트에서는 memtrace.c의 내용은 part 2와 동일하다. 그러나 이번에는 다른 파일인 callinfo.c를 구현하는 것이 관건이다.
\\
callinfo.h를 보면 설명이 나와 있다.
\inputminted[firstline=7,lastline=17, linenos]{C}{../handout/part3/callinfo.h}
이 함수는 인자 3개를 받고 성공시 0, 실패시 음수를 반환해야 한다고 쓰여 있다. 이 인자 3개가 IN용인지 OUT용인지 판단하기 위해 memlog.c를 살펴본다.
\inputminted[firstline=17,lastline=21, linenos]{C}{2019-13674.224747/utils/memlog.c}
callinfo.h의 내용과 다르게 실제 사용하는 측은 리턴값이 -1인지 아닌지를 검사하고 있다. 이 점을 유의하며 callinfo.c의 내용을 작성하였다. 그리고 17행을 보면 \lstinline{get_callinfo}의 인자들은 전부 OUT임을 알 수 있다. 즉, 정보를 \lstinline{get_callinfo} 내부에서 생산하여 호출자에게 전달해 주어야 한다는 것이다.
과제 pdf에 libunwind를 이용하여 stack trace를 출력하는 예제가 있어 이용하였다.

\inputminted[firstline=1, lastline=18,linenos]{C}{2019-13674.224747/part3/callinfo.c}
이 부분은 예제에 따라 libunwind를 사용하기 위한 변수들을 초기화한 것이다. 이제 cursor를 적당히 이동시켜 get\_callinfo를 호출한 mlog를 호출한 우리가 가로챈 함수를 호출한 위치를 찾아야 한다. 따라서 \lstinline{unw_step} 함수를 3번 호출하여 커서를 이동시킨다.
\inputminted[firstline=19, lastline=30,linenos]{C}{2019-13674.224747/part3/callinfo.c}
이제 \lstinline{unw_get_proc_name} 함수를 호출하여 함수 이름과 그 함수의 처음 주소로부터의 오프셋을 구하면 된다.
\inputminted[firstline=34, lastline=36,linenos]{C}{2019-13674.224747/part3/callinfo.c}
이렇게 하고 나니 결과값이 objdump로 예측한 결과와 5 차이가 난다. 그 이유는 libunwind가 알려주는 offset은 함수의 호출 후 돌아갈 return address 즉 함수를 호출하는 \lstinline{callq} 명령 다음 인스트럭션의 주소에 대한 offset을 알려주기 때문이다. 이는 x64 아키텍쳐의 \lstinline{callq} 명령이 스택에 현재 \lstinline{rip} 즉 PC값을 푸시하고 타깃으로 점프하는 식으로 작동하기 때문이다.
\\
objdump에 의한 \lstinline{callq} 명령의 크기는 5바이트이다. 그러므로 \lstinline{unw_get_proc_name}의 결과값에서 5를 빼준 값을 \lstinline{*ofs}에 넣는다.
\inputminted[firstline=37, lastline=40,linenos]{C}{2019-13674.224747/part3/callinfo.c}

\subsection{테스트 결과}
\subsubsection{test1}
\begin{minted}[breaklines]{console}
  user102@SystemProgramming:~/handout/part3$ objdump -d ../test/test1
  (중략)
  Disassembly of section .text:

0000000000400470 <main>:
  400470:       53                      push   %rbx
  400471:       bf 00 04 00 00          mov    $0x400,%edi
  400476:       e8 d5 ff ff ff          callq  400450 <malloc@plt>
  40047b:       bf 20 00 00 00          mov    $0x20,%edi
  400480:       e8 cb ff ff ff          callq  400450 <malloc@plt>
  400485:       bf 01 00 00 00          mov    $0x1,%edi
  40048a:       48 89 c3                mov    %rax,%rbx
  40048d:       e8 be ff ff ff          callq  400450 <malloc@plt>
  400492:       48 89 c7                mov    %rax,%rdi
  400495:       e8 96 ff ff ff          callq  400430 <free@plt>
  40049a:       48 89 df                mov    %rbx,%rdi
  40049d:       e8 8e ff ff ff          callq  400430 <free@plt>
  4004a2:       31 c0                   xor    %eax,%eax
  4004a4:       5b                      pop    %rbx
  4004a5:       c3                      retq
  4004a6:       66 2e 0f 1f 84 00 00    nopw   %cs:0x0(%rax,%rax,1)
  4004ad:       00 00 00
  (후략)
\end{minted}
main:6, main:10, main:1d, main:25, main:2d에서 메모리 할당과 해제 함수가 호출되는 것을 볼 수 있다.

\begin{minted}[breaklines]{console}
  user102@SystemProgramming:~/handout/part3$ make run test1
  cc -I. -I ../utils -o libmemtrace.so -shared -fPIC memtrace.c ../utils/memlog.c ../utils/memlist.c callinfo.c -ldl -lunwind
  [0001] Memory tracer started.
  [0002]         main:6  : malloc( 1024 ) = 0x1286060
  [0003]         main:10 : malloc( 32 ) = 0x12864c0
  [0004]         main:1d : malloc( 1 ) = 0x1286540
  [0005]         main:25 : free( 0x1286540 )
  [0006]         main:2d : free( 0x12864c0 )
  [0007]
  [0008] Statistics
  [0009]   allocated_total      1057
  [0010]   allocated_avg        352
  [0011]   freed_total          33
  [0012]
  [0013] Non-deallocated memory blocks
  [0014]   block              size       ref cnt   caller
  [0015]   0x1286060          1024       1         main:6
  [0016]
  [0017] Memory tracer stopped.
  
\end{minted}
objdump로 예측한 위치와 일치하는 것을 볼 수 있었다.

\subsubsection{test2}
\begin{minted}[breaklines]{console}
  user102@SystemProgramming:~/handout/part3$ objdump -d ../test/test2
  (중략)
  Disassembly of section .text:

0000000000400470 <main>:
  400470:       48 83 ec 08             sub    $0x8,%rsp
  400474:       bf 00 04 00 00          mov    $0x400,%edi
  400479:       e8 d2 ff ff ff          callq  400450 <malloc@plt>
  40047e:       48 89 c7                mov    %rax,%rdi
  400481:       e8 aa ff ff ff          callq  400430 <free@plt>
  400486:       31 c0                   xor    %eax,%eax
  400488:       48 83 c4 08             add    $0x8,%rsp
  40048c:       c3                      retq
  40048d:       0f 1f 00                nopl   (%rax)
  (후략)
\end{minted}
main:9, main:11에서 각각 메모리 할당과 해제가 일어나는 것을 볼 수 있다.
\begin{minted}[breaklines]{console}
  user102@SystemProgramming:~/handout/part3$ make run test2
  cc -I. -I ../utils -o libmemtrace.so -shared -fPIC memtrace.c ../utils/memlog.c ../utils/memlist.c callinfo.c -ldl -lunwind
  [0001] Memory tracer started.
  [0002]         main:9  : malloc( 1024 ) = 0x147a060
  [0003]         main:11 : free( 0x147a060 )
  [0004]
  [0005] Statistics
  [0006]   allocated_total      1024
  [0007]   allocated_avg        1024
  [0008]   freed_total          1024
  [0009]
  [0010] Memory tracer stopped.
\end{minted}
objdump로 예측한 것과 일치한다.

\subsubsection{test3}
\begin{minted}[breaklines]{console}
  user102@SystemProgramming:~/handout/part3$ objdump -d ../test/test3
  (중략)
  Disassembly of section .text:

0000000000400600 <main>:
  400600:       41 54                   push   %r12
  400602:       55                      push   %rbp
  400603:       31 ff                   xor    %edi,%edi
  400605:       53                      push   %rbx
  400606:       48 83 ec 60             sub    $0x60,%rsp
  40060a:       64 48 8b 04 25 28 00    mov    %fs:0x28,%rax
  400611:       00 00
  400613:       48 89 44 24 58          mov    %rax,0x58(%rsp)
  400618:       31 c0                   xor    %eax,%eax
  40061a:       e8 a1 ff ff ff          callq  4005c0 <time@plt>
  40061f:       89 c7                   mov    %eax,%edi
  400621:       48 89 e3                mov    %rsp,%rbx
  400624:       48 8d 6c 24 50          lea    0x50(%rsp),%rbp
  400629:       e8 72 ff ff ff          callq  4005a0 <srand@plt>
  40062e:       eb 15                   jmp    400645 <main+0x45>
  400630:       4c 89 e7                mov    %r12,%rdi
  400633:       48 83 c3 08             add    $0x8,%rbx
  400637:       e8 94 ff ff ff          callq  4005d0 <malloc@plt>
  40063c:       48 89 43 f8             mov    %rax,-0x8(%rbx)
  400640:       48 39 eb                cmp    %rbp,%rbx
  400643:       74 37                   je     40067c <main+0x7c>
  400645:       e8 96 ff ff ff          callq  4005e0 <rand@plt>
  40064a:       99                      cltd
  40064b:       c1 ea 10                shr    $0x10,%edx
  40064e:       8d 34 10                lea    (%rax,%rdx,1),%esi
  400651:       0f b7 f6                movzwl %si,%esi
  400654:       29 d6                   sub    %edx,%esi
  400656:       4c 63 e6                movslq %esi,%r12
  400659:       e8 82 ff ff ff          callq  4005e0 <rand@plt>
  40065e:       a8 01                   test   $0x1,%al
  400660:       75 ce                   jne    400630 <main+0x30>
  400662:       4c 89 e6                mov    %r12,%rsi
  400665:       bf 01 00 00 00          mov    $0x1,%edi
  40066a:       48 83 c3 08             add    $0x8,%rbx
  40066e:       e8 3d ff ff ff          callq  4005b0 <calloc@plt>
  400673:       48 89 43 f8             mov    %rax,-0x8(%rbx)
  400677:       48 39 eb                cmp    %rbp,%rbx
  40067a:       75 c9                   jne    400645 <main+0x45>
  40067c:       48 8d 5c 24 48          lea    0x48(%rsp),%rbx
  400681:       48 8d 6c 24 f8          lea    -0x8(%rsp),%rbp
  400686:       66 2e 0f 1f 84 00 00    nopw   %cs:0x0(%rax,%rax,1)
  40068d:       00 00 00
  400690:       48 8b 3b                mov    (%rbx),%rdi
  400693:       48 83 eb 08             sub    $0x8,%rbx
  400697:       e8 d4 fe ff ff          callq  400570 <free@plt>
  40069c:       48 39 eb                cmp    %rbp,%rbx
  40069f:       75 ef                   jne    400690 <main+0x90>
  4006a1:       31 c0                   xor    %eax,%eax
  4006a3:       48 8b 4c 24 58          mov    0x58(%rsp),%rcx
  4006a8:       64 48 33 0c 25 28 00    xor    %fs:0x28,%rcx
  4006af:       00 00
  4006b1:       75 09                   jne    4006bc <main+0xbc>
  4006b3:       48 83 c4 60             add    $0x60,%rsp
  4006b7:       5b                      pop    %rbx
  4006b8:       5d                      pop    %rbp
  4006b9:       41 5c                   pop    %r12
  4006bb:       c3                      retq
  4006bc:       e8 bf fe ff ff          callq  400580 <__stack_chk_fail@plt>
  4006c1:       66 2e 0f 1f 84 00 00    nopw   %cs:0x0(%rax,%rax,1)
  4006c8:       00 00 00
  4006cb:       0f 1f 44 00 00          nopl   0x0(%rax,%rax,1)
  (후략)
\end{minted}
main:37, main:6e에서 메모리를 할당받고, main:97에서 해제하는 것을 볼 수 있다.
\begin{minted}[breaklines]{console}
  user102@SystemProgramming:~/handout/part3$ make run test3
cc -I. -I ../utils -o libmemtrace.so -shared -fPIC memtrace.c ../utils/memlog.c ../utils/memlist.c callinfo.c -ldl -lunwind
[0001] Memory tracer started.
[0002]         main:6e : calloc( 1 , 47824 ) = 0x25f3060
[0003]         main:6e : calloc( 1 , 11452 ) = 0x25feb90
[0004]         main:37 : malloc( 10845 ) = 0x26018b0
[0005]         main:6e : calloc( 1 , 32498 ) = 0x2604370
[0006]         main:37 : malloc( 29647 ) = 0x260c2c0
[0007]         main:37 : malloc( 31076 ) = 0x26136f0
[0008]         main:6e : calloc( 1 , 30023 ) = 0x261b0b0
[0009]         main:37 : malloc( 28743 ) = 0x2622650
[0010]         main:37 : malloc( 45171 ) = 0x26296f0
[0011]         main:6e : calloc( 1 , 11093 ) = 0x26347c0
[0012]         main:97 : free( 0x26347c0 )
[0013]         main:97 : free( 0x26296f0 )
[0014]         main:97 : free( 0x2622650 )
[0015]         main:97 : free( 0x261b0b0 )
[0016]         main:97 : free( 0x26136f0 )
[0017]         main:97 : free( 0x260c2c0 )
[0018]         main:97 : free( 0x2604370 )
[0019]         main:97 : free( 0x26018b0 )
[0020]         main:97 : free( 0x25feb90 )
[0021]         main:97 : free( 0x25f3060 )
[0022]
[0023] Statistics
[0024]   allocated_total      278372
[0025]   allocated_avg        27837
[0026]   freed_total          278372
[0027]
[0028] Memory tracer stopped.
\end{minted}
objdump의 결과와 마찬가지로 \lstinline{malloc}과 \lstinline{calloc}, \lstinline{free}의 호출을 제대로 표시하는 것을 볼 수 있다.
\subsubsection{test4}
\begin{minted}[breaklines]{console}
  user102@SystemProgramming:~/handout/part3$ objdump -d ../test/test4
  (중략)
  0000000000400470 <main>:
  400470:       53                      push   %rbx
  400471:       bf 00 04 00 00          mov    $0x400,%edi
  400476:       e8 d5 ff ff ff          callq  400450 <malloc@plt>
  40047b:       48 89 c3                mov    %rax,%rbx
  40047e:       48 89 c7                mov    %rax,%rdi
  400481:       e8 aa ff ff ff          callq  400430 <free@plt>
  400486:       48 89 df                mov    %rbx,%rdi
  400489:       e8 a2 ff ff ff          callq  400430 <free@plt>
  40048e:       bf 90 6e 70 01          mov    $0x1706e90,%edi
  400493:       e8 98 ff ff ff          callq  400430 <free@plt>
  400498:       31 c0                   xor    %eax,%eax
  40049a:       5b                      pop    %rbx
  40049b:       c3                      retq
  40049c:       0f 1f 40 00             nopl   0x0(%rax)
\end{minted}
main:6, main:11, main:19, main:23 에서 메모리를 할당받고 세 번 해제하는 것을 볼 수 있다.

\begin{minted}[breaklines]{console}
  user102@SystemProgramming:~/handout/part3$ make run test4
cc -I. -I ../utils -o libmemtrace.so -shared -fPIC memtrace.c ../utils/memlog.c ../utils/memlist.c callinfo.c -ldl -lunwind
[0001] Memory tracer started.
[0002]         main:6  : malloc( 1024 ) = 0x969060
[0003]         main:11 : free( 0x969060 )
[0004]         main:19 : free( 0x969060 )
*** Error in `../test/test4': double free or corruption (!prev): 0x0000000000969060 ***
[0005]      realloc:2e95: malloc( 36 ) = 0x7fc2cc0008c0
[0006]           (nil) : calloc( 1182 , 1 ) = 0x7fc2cc000940
[0007]           (nil) : malloc( 36 ) = 0x7fc2cc000e40
[0008]           (nil) : malloc( 56 ) = 0x7fc2cc000ec0
[0009] _dl_debug_state:1059: calloc( 15 , 24 ) = 0x7fc2cc000f50
======= Backtrace: =========
/lib/x86_64-linux-gnu/libc.so.6(+0x777e5)[0x7fc2d30257e5]
/lib/x86_64-linux-gnu/libc.so.6(+0x8037a)[0x7fc2d302e37a]
/lib/x86_64-linux-gnu/libc.so.6(cfree+0x4c)[0x7fc2d303253c]
./libmemtrace.so(free+0x6c)[0x7fc2d3378feb]
../test/test4[0x40048e]
/lib/x86_64-linux-gnu/libc.so.6(__libc_start_main+0xf0)[0x7fc2d2fce830]
../test/test4[0x4004c9]
======= Memory map: ========
00400000-00401000 r-xp 00000000 ca:01 657555                             /home/user102/handout/test/test4
00600000-00601000 r--p 00000000 ca:01 657555                             /home/user102/handout/test/test4
00601000-00602000 rw-p 00001000 ca:01 657555                             /home/user102/handout/test/test4
00969000-0098a000 rw-p 00000000 00:00 0                                  [heap]
7fc2cc000000-7fc2cc021000 rw-p 00000000 00:00 0
7fc2cc021000-7fc2d0000000 ---p 00000000 00:00 0
7fc2d297c000-7fc2d2992000 r-xp 00000000 ca:01 2097679                    /lib/x86_64-linux-gnu/libgcc_s.so.1
7fc2d2992000-7fc2d2b91000 ---p 00016000 ca:01 2097679                    /lib/x86_64-linux-gnu/libgcc_s.so.1
7fc2d2b91000-7fc2d2b92000 rw-p 00015000 ca:01 2097679                    /lib/x86_64-linux-gnu/libgcc_s.so.1
7fc2d2b92000-7fc2d2b9e000 r-xp 00000000 ca:01 1453561                    /usr/local/lib/libunwind.so.8.0.1
7fc2d2b9e000-7fc2d2d9e000 ---p 0000c000 ca:01 1453561                    /usr/local/lib/libunwind.so.8.0.1
7fc2d2d9e000-7fc2d2d9f000 r--p 0000c000 ca:01 1453561                    /usr/local/lib/libunwind.so.8.0.1
7fc2d2d9f000-7fc2d2da0000 rw-p 0000d000 ca:01 1453561                    /usr/local/lib/libunwind.so.8.0.1
7fc2d2da0000-7fc2d2daa000 rw-p 00000000 00:00 0
7fc2d2daa000-7fc2d2dad000 r-xp 00000000 ca:01 2097667                    /lib/x86_64-linux-gnu/libdl-2.23.so
(중략)
7fc2d3372000-7fc2d3374000 rw-p 001c4000 ca:01 2097653                    /lib/x86_64-linux-gnu/libc-2.23.so
7fc2d3374000-7fc2d3378000 rw-p 00000000 00:00 0
7fc2d3378000-7fc2d337b000 r-xp 00000000 ca:01 657537                     /home/user102/handout/part3/libmemtrace.so
7fc2d337b000-7fc2d357a000 ---p 00003000 ca:01 657537                     /home/user102/handout/part3/libmemtrace.so
7fc2d357a000-7fc2d357b000 r--p 00002000 ca:01 657537                     /home/user102/handout/part3/libmemtrace.so
7fc2d357b000-7fc2d357c000 rw-p 00003000 ca:01 657537                     /home/user102/handout/part3/libmemtrace.so
7fc2d357c000-7fc2d35a2000 r-xp 00000000 ca:01 2097629                    /lib/x86_64-linux-gnu/ld-2.23.so
7fc2d3797000-7fc2d379a000 rw-p 00000000 00:00 0
7fc2d379d000-7fc2d37a1000 rw-p 00000000 00:00 0
7fc2d37a1000-7fc2d37a2000 r--p 00025000 ca:01 2097629                    /lib/x86_64-linux-gnu/ld-2.23.so
7fc2d37a2000-7fc2d37a3000 rw-p 00026000 ca:01 2097629                    /lib/x86_64-linux-gnu/ld-2.23.so
7fc2d37a3000-7fc2d37a4000 rw-p 00000000 00:00 0
7ffe44d79000-7ffe44d9a000 rw-p 00000000 00:00 0                          [stack]
7ffe44de2000-7ffe44de5000 r--p 00000000 00:00 0                          [vvar]
7ffe44de5000-7ffe44de7000 r-xp 00000000 00:00 0                          [vdso]
ffffffffff600000-ffffffffff601000 r-xp 00000000 00:00 0                  [vsyscall]
Aborted (core dumped)
Makefile:37: recipe for target 'run' failed
make: *** [run] Error 134
\end{minted}
아직 part1, 2와 마찬가지로 illegal free와 double free로  크래시되는 것을 볼 수 있다.

\subsubsection{test5}
\begin{minted}[breaklines]{console}
user102@SystemProgramming:~/handout/part3$ objdump -d ../test/test5
(중략)
00000000004004c0 <main>:
  4004c0:       48 83 ec 08             sub    $0x8,%rsp
  4004c4:       bf 0a 00 00 00          mov    $0xa,%edi
  4004c9:       e8 c2 ff ff ff          callq  400490 <malloc@plt>
  4004ce:       be 64 00 00 00          mov    $0x64,%esi
  4004d3:       48 89 c7                mov    %rax,%rdi
  4004d6:       e8 c5 ff ff ff          callq  4004a0 <realloc@plt>
  4004db:       be e8 03 00 00          mov    $0x3e8,%esi
  4004e0:       48 89 c7                mov    %rax,%rdi
  4004e3:       e8 b8 ff ff ff          callq  4004a0 <realloc@plt>
  4004e8:       be 10 27 00 00          mov    $0x2710,%esi
  4004ed:       48 89 c7                mov    %rax,%rdi
  4004f0:       e8 ab ff ff ff          callq  4004a0 <realloc@plt>
  4004f5:       be a0 86 01 00          mov    $0x186a0,%esi
  4004fa:       48 89 c7                mov    %rax,%rdi
  4004fd:       e8 9e ff ff ff          callq  4004a0 <realloc@plt>
  400502:       48 89 c7                mov    %rax,%rdi
  400505:       e8 66 ff ff ff          callq  400470 <free@plt>
  40050a:       31 c0                   xor    %eax,%eax
  40050c:       48 83 c4 08             add    $0x8,%rsp
  400510:       c3                      retq
  400511:       66 2e 0f 1f 84 00 00    nopw   %cs:0x0(%rax,%rax,1)
  400518:       00 00 00
  40051b:       0f 1f 44 00 00          nopl   0x0(%rax,%rax,1)
  (후략)

\end{minted}
main:9, main:16, main:23, main:30, main:3d, main:45에서 메모리를 할당받고 해제하는 것을 볼 수 있다.

\begin{minted}[breaklines]{console}
  user102@SystemProgramming:~/handout/part3$ make run test5
cc -I. -I ../utils -o libmemtrace.so -shared -fPIC memtrace.c ../utils/memlog.c ../utils/memlist.c callinfo.c -ldl -lunwind
[0001] Memory tracer started.
[0002]         main:9  : malloc( 10 ) = 0x176c060
[0003]         main:16 : realloc( 0x176c060 , 100 ) = 0x176c0d0
[0004]         main:23 : realloc( 0x176c0d0 , 1000 ) = 0x176c190
[0005]         main:30 : realloc( 0x176c190 , 10000 ) = 0x176c5d0
[0006]         main:3d : realloc( 0x176c5d0 , 100000 ) = 0x176c5d0
[0007]         main:45 : free( 0x176c5d0 )
[0008]
[0009] Statistics
[0010]   allocated_total      111110
[0011]   allocated_avg        22222
[0012]   freed_total          111110
[0013]
[0014] Memory tracer stopped.
\end{minted}
objdump의 결과와 일치한다.

\section{Bonus: Detect and Ignore Illegal Deallocations}
Bonus에서는 part 1~3을 test4에 대해 적용했을 때 double free와 illegal free 시 크래시되는 것을 예방하고 대신 각각 DOUBLE FREE와 ILLEGAL FREE 정보를 출력하도록 한다. 이번에는 \lstinline{free}와 \lstinline{realloc} 에 처리가 추가된다. 기존 \lstinline{list}에서 현재 free 하려는 메모리 영역에 대한 item을 검색하고, 존재하지 않을 경우 Illegal free 오류를, 존재는 하나 이미 reference count가 0일 경우 Double free 오류를 출력하고 무시하는 코드가 추가되었다.

\inputminted[firstline=45, lastline=60,linenos]{C}{2019-13674.224747/bonus/memtrace.c}
\lstinline{free}에서는 위의 내용이 그대로 구현되어 있다.

\inputminted[firstline=74, lastline=108,linenos]{C}{2019-13674.224747/bonus/memtrace.c}
\lstinline{realloc}에서는 처리가 좀 더 복잡한데, 그 이유는 etl에 올라온 realloc 예시 출력을 보면 \lstinline{realloc} 호출 인자와 결과값을 먼저 출력하고 나서 illegal free나 double free 오류를 출력하기 때문이다. 그렇다고 illegal free나 double free 검출보다 realloc을 먼저 할 수는 없을 것이므로, 판단 결과를 저장하는 변수인 \lstinline{double_free} 변수와 \lstinline{illegal_free} 변수를 사용하였다.

\subsection{테스트 결과}

test1~test3, test5는 앞의 part3의 테스트 결과와 같을 것이므로 part1~part3에서 크래시되었던 test4에 대해 테스트를 해 본다.
\subsubsection{test4}
\inputminted[firstline=5, lastline=10,linenos]{C}{2019-13674.224747/test/test4.c}
double free와 illegal free가 일어남을 볼 수 있다.
\begin{minted}[breaklines]{console}
  user102@SystemProgramming:~/handout/bonus$ make run test4
  cc -I. -I ../utils -o libmemtrace.so -shared -fPIC memtrace.c ../utils/memlog.c ../utils/memlist.c callinfo.c -ldl -lunwind
  [0001] Memory tracer started.
  [0002]         main:6  : malloc( 1024 ) = 0x12d8060
  [0003]         main:11 : free( 0x12d8060 )
  [0004]         main:19 : free( 0x12d8060 )
  [0005]             *** DOUBLE_FREE  *** (ignoring)
  [0006]         main:23 : free( 0x1706e90 )
  [0007]             *** ILLEGAL_FREE *** (ignoring)
  [0008]
  [0009] Statistics
  [0010]   allocated_total      1024
  [0011]   allocated_avg        1024
  [0012]   freed_total          1024
  [0013]
  [0014] Memory tracer stopped.
\end{minted}
double free와 illegal free를 발견하고 무시하여 크래시를 예방하며 로그를 잘 출력하는 것을 볼 수 있다.
\section{Conclusion}
\subsection{어려웠던 점}
\subsubsection{익숙한 환경의 부재}
작년 수업에서 처음 배워 사용하던 리눅스 텍스트 기반 에디터인 emacs 에디터가 없어서 \lstinline{i}와 \lstinline{:wq}밖에 모르던 vi 에디터를 사용하였는데, vi 환경에 적응하는 데 시간이 걸렸다. 다행히 검색을 통해 필요한 기능들을 익힐 수 있었다.

\subsubsection{libunwind}
Part 3에서 libunwind를 이용해 메모리 할당 함수들의 호출 위치를 추적하는데, objdump로 예측했던 offset과 libunwind가 \lstinline{unw_get_proc_name}으로 알려주는 offset이 서로 달라 혼란이 있었다. 기본적으로 libunwind도 스택에 저장된 리턴 어드레스를 이용하는 것이라 해서 callq의 인스트럭션 크기만큼 보정을 해 주어야 할 것 같다는 생각은 들었지만, 예제들을 아무리 찾아 봐도 보정에 대한 이야기는 없어서 혼란스러웠다. 결국 그 추측에 확신을 얻기 위해 libunwind의 소스 코드를 찾아 보았는데, 파일이 너무 많아서 어려움을 겪었다. 다행히 질의응답에 누군가 질문을 올린 것을 발견하여 5의 보정값을 빼는 것에 확신을 얻을 수 있었다.
\subsection{놀라웠던 점}
\subsubsection{library interpositioning의 간단함}
이렇게 간단한 코드로 어떤 프로그램의 메모리 사용을 해당 프로그램을 수정하지 않고 추적할 수 있다는 점이 놀라웠다. 또한 라이브러리를 조금 이용하여 실제 함수 호출 위치와 그 호출자 함수의 이름 등 유용한 정보들을 손쉽게 얻을 수 있다는 사실이 놀라웠다.
\subsubsection{call이 없는 main과 최적화}
objdump로 test 들을 분석해 보는 과정에서 testx도 분석해 보았는데, testx의 \lstinline{main} 코드를 보면 \lstinline{foo} 를 호출하는 부분이 전혀 없었다. Makefile을 살펴보아 -O2 플래그를 확인하였고, 컴파일러 최적화 과정의 함수 인라이닝 때문이라는 것을 알게 되었다.
\\
\lstinline{foo} 함수는 \lstinline{main}에 인라인 처리가 되었음에도 불구하고 바이너리 파일 안에 존재했는데, \lstinline{static}이 붙지 않아 심볼이 남았기 때문이라고 생각할 수 있다.
\\
\lstinline{main} 함수에도 마지막 함수 호출은 \lstinline{jmp} 인스트럭션으로 대체되어 있고 제대로 된 \lstinline{return}이 없던 것을 볼 수 있었는데, 이것도 컴파일러 최적화의 결과이며, \lstinline{call} 후 타깃 함수의 \lstinline{ret}, 그 후 \lstinline{main}의 \lstinline{ret}을 간소화한 것으로 생각이 든다. 이와 더불어 스택에 대한 걱정을 잠시 했었는데(원래는 리턴 어드레스 푸시로 인해 스택이 약간 변화하므로), 이 시스템은 64비트라 적은 매개변수는 레지스터로 전달하여 상관이 없구나 하는 생각이 들었다. 그에 이어서 든 생각인데, 매개변수도 쉽게 레지스터에 전달하는데 리턴 어드레스도 ARM 프로세서와 같이 새로운 레지스터에 저장하는 것은 어떨까 하는 생각도 잠깐 들었다.

\subsubsection{\lstinline{hlt} 인스트럭션}
레포트를 작성할 때 objdump의 출력을 복사하다가 우연히 \lstinline{_start}함수에 \lstinline{hlt} 인스트럭션이 존재하는 것을 발견하였다.
\begin{minted}[breaklines]{asm}
  00000000004004a0 <_start>:
  4004a0:       31 ed                   xor    %ebp,%ebp
  4004a2:       49 89 d1                mov    %rdx,%r9
  4004a5:       5e                      pop    %rsi
  4004a6:       48 89 e2                mov    %rsp,%rdx
  4004a9:       48 83 e4 f0             and    $0xfffffffffffffff0,%rsp
  4004ad:       50                      push   %rax
  4004ae:       54                      push   %rsp
  4004af:       49 c7 c0 10 06 40 00    mov    $0x400610,%r8
  4004b6:       48 c7 c1 a0 05 40 00    mov    $0x4005a0,%rcx
  4004bd:       48 c7 c7 70 04 40 00    mov    $0x400470,%rdi
  4004c4:       e8 77 ff ff ff          callq  400440 <__libc_start_main@plt>
  4004c9:       f4                      hlt
  4004ca:       66 0f 1f 44 00 00       nopw   0x0(%rax,%rax,1)
\end{minted}

분명 알고 있는 바로는 \lstinline{hlt} 인스트럭션은 ring 0 레벨에서만 실행 가능한 것으로 알고 있는데, 바로 다음의 \lstinline{nopw 0x0(%rax, %rax, 1)}과 같은 단순한 패딩은 아닌 것 같아서 검색해 보았다. \href{https://github.com/bminor/glibc/blob/2a69f853c03034c2e383e0f9c35b5402ce8b5473/sysdeps/i386/start.S#L118}{Github}의 코드를 찾아보니 이것은 프로그램의 크래시를 의도한 것이었다.
\begin{minted}[breaklines]{asm}
  hlt			/* Crash if somehow `exit' does return.  */
\end{minted}
이 \lstinline{hlt} 명령은 단지 General Protection Fault를 이용해 프로그램을 종료시키기 위한 것이었다. exit()함수가 실패했으므로 확실하게 프로그램을 종료시키기 위한 것이었다.
\end{document}