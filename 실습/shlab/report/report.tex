\documentclass{report}
\usepackage{graphicx}
\usepackage{kotex}
\usepackage{lipsum}
\usepackage{listings}
\usepackage{graphicx}
\usepackage[a4paper, total={6in, 10in}]{geometry}
\usepackage{float}
\usepackage{xcolor}
\usepackage{color}
\usepackage{minted}
\usepackage{hyperref}
\renewcommand{\thesection}{\arabic{section}}

\title{시스템 프로그래밍 과제 2 shlab 레포트 \\
\large Shell Lab: Writing Your Own Unix Shell}
\date{}
\author{
  2019-13674\\
  양현서 \\
  바이오시스템소재학부\\
}
\lstset{
    basicstyle=\ttfamily\small,
    numberstyle=\footnotesize,
    numbers=left,
    frame=single,
    tabsize=2,
    title=\lstname,
    keywordstyle={\color{blue}}
    escapeinside={\%*}{*)},
    breaklines=true,
    breakatwhitespace=true,
    framextopmargin=2pt,
    framexbottommargin=2pt,
    extendedchars=false,
    inputencoding=utf8
}
\hypersetup{
    colorlinks,
    linkcolor={red!50!black},
    citecolor={blue!50!black},
    urlcolor={blue!80!black}
}

\makeatletter
\AtBeginEnvironment{minted}{\dontdofcolorbox}
\def\dontdofcolorbox{\renewcommand\fcolorbox[4][]{##4}}
\makeatother

\begin{document}
\maketitle
\tableofcontents

\section{Introduction}
\paragraph{}
 수업 시간에 signal과 프로세스 제어 등을 배웠다. 프로세스의 실행 중에는 강제 종료나 일시 정지 등 여러 가지 특별한 상황이 발생할 수 있다. 그리고 시그널을 통해 프로그램의 실행 중 발생한 이러한 상황들을 처리할 수 있다. 이번 랩에서는 실제 Unix 셸처럼 하위 프로세스들을 실행하거나, bg, fg, job 명령과 \lstinline{^Z}, \lstinline{^I} 등 시그널 입력으로 그러한 프로세스들을 관리할 수 있고, 발생하는 시그널들에 제대로 반응하는 미니 셸을 만든다.

\section{Step 1: 기본 뼈대 완성}
Part 1에서는 명령어 입력을 처리하고 quit, bg, fg, job 등의 기초 커맨드를 처리하는 부분을 작성한다.
\inputminted[firstline=293,lastline=306, linenos]{C}{../shlab/tsh.c}
\lstinline{argv[0]}을 기준으로 명령을 판별하여 해당하는 기능들을 실행한다. builtin command의 경우는 \lstinline{eval}에 커맨드가 처리되었음을 알리기 위해 1을 리턴, 아닌 경우에는 0을 리턴한다.

그다음은 이 부분의 핵심인 \lstinline{eval}을 설명한다.

\inputminted[firstline=167,lastline=174, linenos]{C}{../shlab/tsh.c}
\lstinline{parseline}을 호출하여 argv들을 얻어오고, 해당 명령이 background로 실행해야 하는지를 알아낸다.
인자로 아무 것도 들어오지 않으면 아무 것도 하지 않는다.

\inputminted[firstline=175,lastline=178, linenos]{C}{../shlab/tsh.c}

명령이 builtin command라면 처리하고 바로 종료한다.

\inputminted[firstline=179,lastline=196, linenos]{C}{../shlab/tsh.c}
\lstinline{fork} 도중 문제를 막기 위해 \lstinline{SIGCHLD, SIGINT, SIGTSTP}을 블록한다. \lstinline{sigemptyset}을 통해 구조체를 초기화하고, \lstinline{sigaddset}을 통해 시그널 집합에 \lstinline{SIGCHLD, SIGINT, SIGTSTP}을 등록한다. 그다음 \lstinline{sigprocmask}를 이용하여 잠시 시그널들을 블록한 것이다.

\inputminted[firstline=197,lastline=199, linenos]{C}{../shlab/tsh.c}
\lstinline{fork}가 실패하면 오류를 출력하고 종료한다.

\inputminted[firstline=200,lastline=215, linenos]{C}{../shlab/tsh.c}
\lstinline{fork}의 리턴 값이 0인 경우 child process에서 실행중인 것이다. 우선 \lstinline{fork}전에 블록했던 시그널을 언블록한 후, \lstinline{execve}를 이용하여 대상을 실행시킨다. 성공적으로 수행하였다면 \lstinline{execve}는 리턴하지 않는다. 그러므로 리턴한다면 에러를 출력하고 종료한다.

\inputminted[firstline=216,lastline=230, linenos]{C}{../shlab/tsh.c}
\lstinline{fork}의 리턴 값이 양수인 경우 parent process에서 실행중인 것이다. 새로운 job을 리스트에 등록하고, \lstinline{fork}전에 블록했던 시그널을 언블록한 후, \lstinline{bg} 여부에 따라 \lstinline{waitfg}를 호출하여 이번 job을 기다리거나, 표준 출력에 새로 생긴 background job에 대한 정보를 출력한 후 함수를 종료한다.

\section{Step 2: bg와 fg 명령 처리 실제 구현}
셸에서, \lstinline{bg/fg} [인자] 를 입력하면 해당 인자(pid 또는 jobid)에 해당하는 job을 찾아 background/foreground 작업으로 변경해 재생시킨다.
\inputminted[firstline=311,lastline=346, linenos]{C}{../shlab/tsh.c}
bg/fg 명령의 인자의 첫 글자가 \lstinline{'%'}인지에 따라 \%라면 job id, \%가 아니라면 pid로 해석하여 \lstinline{strtol}함수로 정수형으로 변환한 뒤, 이것을 키로 이용하여 해당하는 \lstinline{struct job_t *} 포인터를 얻어온다. 실패할 경우 그에 해당하는 오류 메시지를 출력한다.

\inputminted[firstline=347,lastline=368, linenos]{C}{../shlab/tsh.c}
이어 명령이 \lstinline{bg} 라면 job의 상태를 \lstinline{BG}, \lstinline{fg}라면 \lstinline{FG}로 바꾼다. 그리고 bg, fg 명령이 들어왔다는 것은 이미 그 job이 background로 돌고 있거나 멈춰 있다는 것이므로 작업을 재개하기 위해 \lstinline{kill} 함수를 통해 해당 job의 pid 그룹에 대해 \lstinline{SIGCONT} 시그널을 보낸다. 그 다음 명령이 bg라면 화면에 background job에 대한 정보를 출력하고 fg라면 \lstinline{waitfg} 함수를 호출하여 foreground job의 종료를 기다린다.

\section{Step 3: 시그널 처리와 waitfg 구현}
이번에는 셸에서 생성한 자식 프로세스에게 시그널을 제대로 전달하거나 자식 프로세스가 종료될 때까지 대기하는 매커니즘을 작성한다. 과제 handout에 힌트가 제시되어 있어서 도움이 되었다.
자식 프로세스가 종료되면 부모 프로세스에게 \lstinline{SIGCHLD} 시그널이 전달된다. 이것을 이용하여 \lstinline{sigchld_handler}를 만들고, \lstinline{waitfg}를 구현할 수 있다.
\inputminted[firstline=388,lastline=415, linenos, breaklines]{C}{../shlab/tsh.c}
\lstinline{SIGCHLD} 시그널을 받았을 때 실행되는 함수 \lstinline{sigchld_handler}에서는 모든 자식 프로세스들에 대해 \lstinline{waitpid}를 수행하는데, \lstinline{WNOHANG} 옵션을 이용하여 당장 처리할 자식 프로세스가 더 없으면 0을 리턴받아 루프를 종료하게 하였다. 또한 \lstinline{waitpid} 함수의 설명에 따르면 이 함수로 자식 프로세스들을 reap 하여 좀비 프로세스들을 처리할 수도 있다고 하니 좀비 자식 프로세스들을 처리할 수 있다. \lstinline{WUNTRACED}를 이용하여 해당 자식 프로세스의 상태가 stopped인지 판단할 수 있는 매크로 \lstinline{WIFSTOPPED(status)}을 이용할 수 있었다.
위와 같이 \lstinline{sigchld_handler}에서 foreground job의 \lstinline{state}를 \lstinline{ST}(정지) 또는 \lstinline{UNDEF}(\lstinline{deletejob -> clearjob})으로 바꿔주게 설계하고, 아래와 같이 \lstinline{waitfg}를 구현하였다.
\inputminted[firstline=370,lastline=382, linenos, breaklines]{C}{../shlab/tsh.c}
pid에서 해당하는 job의 구조체를 얻어오고, 해당 job의 \lstinline{state}가 \lstinline{FG}인 동안 1초씩 대기한다. 이렇게 하면 해당 job이 kill되거나 fg가 아닌 경우가 될 때까지 대기하는 것이 가능해진다.

다음은 나머지 시그널들의 핸들러이다. foregrund에 있는 프로세스 그룹들에 해당 시그널을 전달해 준다. 
\inputminted[firstline=417,lastline=445, linenos, breaklines]{C}{../shlab/tsh.c}
\lstinline{SIGINT}와 \lstinline{SIGTSTP} 핸들러는 현재 foreground job으로 관리하고 있는 job의 프로세스 그룹에게만 해당 시그널을 전달해 준다. 앞의 \lstinline{setpgid}를 이용하여 이 셸에게만 시그널이 전달되게 하고, 그것을 내부적으로 관리한 foreground process에만 전달하는 것이다. \lstinline{kill}에 pid를 주면 그 프로세스에만 전달되므로 \lstinline{-pid}를 주어 그 프로세스와 그 프로세스의 자식 프로세스들에까지 전달되게 하였다.

\section{Conclusion}
\subsection{어려웠던 점}
\subsubsection{개발 방향 탐색}
지난번 \lstinline{linklab}에서는 step 1, 2, 3을 단계적으로 해결하면 결과물이 완성되는 것이었지만 이번 \lstinline{shlab}은 그런 단계 구분은 없고 뼈대 코드를 이용하여 알아서 설계하고 완성하는 것이라 조금 더 어려웠다.

\subsection{놀라웠던 점}
\subsubsection{간단함}
생각해 주어야 할 부분만 잘 처리해 주니 생각보다 구현할 내용이 많지 않아서 놀라웠다.
\subsubsection{\lstinline{SIGSTP}과 \lstinline{SIGTSTP}의 차이}
\lstinline{SIGSTP}은 프로그램에서 발생시키는 것이고, \lstinline{SIGTSTP}은 터미널에서 발생시키는 것이다. 또한 \lstinline{SIGTSTP}은 무시할 수 있지만, \lstinline{SIGSTP}은 무시할 수 없다고 한다.
\end{document}